\documentclass[12pt]{amsart}
\bibliographystyle{amsalpha}

\usepackage{times,color,amsmath,amsthm,amssymb,amscd,epsfig,latexsym,graphicx,
mathtools,array,eucal,layout,fancyhdr,marginnote,url,enumitem,mathrsfs}
\usepackage[lite]{mtpro2}
\usepackage{hyperref}
\usepackage[all]{xy}
\usepackage[percent]{overpic}

\usepackage[protrusion=true,expansion=true]{microtype}

\setcounter{tocdepth}{1}
\setlength\textheight{8in}
\setlength\textwidth{6in}
\setlength\oddsidemargin{0.25in}
\setlength\evensidemargin{0.25in}

\newcommand{\ds}{\displaystyle}
\newcommand{\diam}{\operatorname{diam}}
\newcommand{\area}{\operatorname{area}}
\newcommand{\dist}{\operatorname{dist}}
\newcommand{\len}{\operatorname{length}}
\newcommand{\ord}{\operatorname{ord}}
\newcommand{\myint}{\operatorname{int}}
\newcommand{\Aut}{\operatorname{Aut}}
\newcommand{\wtl}{\widetilde}
\newcommand{\wht}{\widehat}
\newcommand{\ve}{\varepsilon}
\newcommand{\es}{\emptyset}
\newcommand{\sm}{\smallsetminus}
\newcommand{\bd}{\partial}
\newcommand{\Chat}{\widehat{\Bbb C}}
\newcommand{\myre}{\operatorname{Re}}
\newcommand{\myim}{\operatorname{Im}}
\newcommand{\ov}{\overline}
\newcommand{\io}{\iota}
\newcommand{\con}{\operatorname{const.}}
\newcommand{\res}{\operatorname{res}}
\newcommand{\OO}{{\mathcal O}}
\newcommand{\MM}{{\mathcal M}}
\newcommand{\CC}{{\mathbb C}}
\newcommand{\PP}{{\mathbb P}}
\newcommand{\RR}{{\mathbb R}}
\newcommand{\HH}{{\mathbb H}}
\newcommand{\TT}{{\mathbb T}}
\newcommand{\II}{{\mathbb I}}
\newcommand{\ZZ}{{\mathbb Z}}
\newcommand{\NN}{{\mathbb N}}
\newcommand{\DD}{{\mathbb D}}
\newcommand{\QQ}{{\mathbb Q}}
\newcommand{\vs}{\vspace{2mm}}
\newcommand{\vr}{\varrho}
\newcommand{\wind}{{\sf W}}
\newcommand{\iso}{\stackrel{\cong}{\longrightarrow}}
\newcommand{\Cbar}{\overline{\mathbb  C}}
\newcommand{\Cstar}{{\mathbb  C}^*}
\newcommand{\Dstar}{{\mathbb  D}^*}
\newcommand{\Sen}{{\mathbb  S}^1}

%Discourage hyphenation:
\hyphenpenalty=5000 \tolerance=1000

\thispagestyle{empty}

\begin{document}
\begin{center}
{\bf \large Math 704 Problem Set 6 Solutions} \vs 
\end{center}

\noindent
{\bf Problem 1.} Suppose $f \in \OO(\DD)$ and the sequence $\{ f^{(n)}(0) \}_{n \geq 1}$ grows at most exponentially fast, i.e., there is a constant $\lambda>1$ such that $|f^{(n)}(0)|<\lambda^n$ for all $n \geq 1$. Show that $f$ extends to an entire function. \vs \\
Let $f(z)=\sum_{n=0}^\infty a_n \, z^n$ for $|z|<1$. By the assumption,
$$
|a_n| = \frac{|f^{(n)}(0)|}{n!} \leq \frac{\lambda^n}{n!} \Longrightarrow 0 \leq |a_n|^{1/n} \leq \frac{\lambda}{(n!)^{1/n}} 
$$  
for all $n$. Since $(n!)^{1/n}$ is easily seen to tend to infinity as $n \to \infty$, it follows that $\lim_{n \to \infty} |a_n|^{1/n} =0$. Thus, the radius of convergence $R=1/\lim_{n \to \infty} |a_n|^{1/n}$ of the power series of $f$ is $+\infty$. As such, this power series provides an extension of $f$ to an entire function. \\

\noindent
{\bf Problem 2.} Let $f$ be a holomorphic function defined in a neighborhood of the origin, say $\DD(0,r)$, which satisfies
$$
f(2z)=(f(z))^2 \qquad \text{whenever} \ |z|<r.
$$
Use this functional equation to show that $f$ can be extended to an entire function. Can you determine all such entire functions explicitly? \vs \\
Take $z \in \CC$ and find the smallest integer $n \geq 0$ such that $|z|/2^n<r$. Set  
\begin{equation}\label{yek}
F(z) = \Big( f \Big( \frac{z}{2^n} \Big) \Big)^{2^n}.
\end{equation}
Observe that the right side of \eqref{yek} remains unchanged if you replace $n$ by any integer greater than $n$. In fact, if $|z|/2^k<r$, then 
$f(z/2^k)=f(2z/2^{k+1})=(f(z/2^{k+1}))^2$ so $(f(z/2^k))^{2^k}=(f(z/2^{k+1}))^{2^{k+1}}$. \vs

Thus, we have a well-defined function $F: \CC \to \CC$ which is holomorphic by \eqref{yek} (if $n$ works for some $z$, the same $n$ works for all points sufficiently close to $z$). Moreover, $F(z)=f(z)$ for $|z|<r$ since we can take $n=0$ in this case. Note that by \eqref{yek} the functional equation $F(2z)=(F(z))^2$ still holds for all $z \in \CC$. In particular, $F(0)=(F(0))^2$, so $F(0)$ is either $0$ or $1$. \vs

To find all such $F$, we consider two cases: \vs

{\it Case 1.} $F(0)=0$. In this case $F$ must be  identically zero. Otherwise, let $m=\ord(F,0)=\ord(f,0) \geq 1$ and observe that in the equation $F(2z)=(F(z))^2$ the left side has a zero of order $m$ at the origin while the right side has a zero of order $2m$ at the origin. Contradiction! \vs

{\it Case 2.} $F(0)=1$. In this case $F$ has no zeros in $\CC$. In fact, if $F(p)=0$ for some $p$ (necessarily $p \neq 0$), then $(F(p/2^n))^{2^n}=F(p)=0$ so $F(p/2^n)=0$ for all $n \geq 0$. By continuity, this implies $F(0)=0$, which is a contradiction. Now $F$, being a non-vanishing entire function, must be of the form $F=\exp(G)$ for some $G \in \OO(\CC)$ with $G(0)=0$. Since 
$$
\exp(G(2z))=F(2z)=(F(z))^2=\exp(2G(z)) \qquad \text{for all} \ z \in \CC,
$$       
we have $G(2z)-2G(z)=2\pi i n$ for an integer $n$ (independent of $z$). In view of $G(0)=0$, we must have $n=0$. Writing $G(z)=\sum a_k \, z^k$ and imposing the equation $G(2z)=2G(z)$ then shows that $\sum a_k \, 2^k z^k = 2 \sum a_k \, z^k$ for all $z$, so $2^k a_k = 2 a_k$ for all $k \geq 0$. This gives $a_k=0$ for all $k \geq 0$ other than $k=1$. Thus, $G$ is linear of the form $G(z)=a_1z$ and $F(z)=\exp (a_1z)$ for an arbitrary $a_1 \in \CC$. \\ 

\noindent
{\bf Problem 3.} The power series $f(z)=\sum_{n=0}^{\infty} z^{2^n}=z+z^2+z^4+z^8+\cdots$ has radius of convergence $1$, so $f \in \OO(\DD)$. By Hadamard's gap theorem, $\TT$ is the natural boundary of $f$. Verify this directly by showing that $\lim_{r \to 1} f(re^{2\pi i t})=\infty$ for every dyadic rational $t$, i.e., those of the form $t=a/2^b$ for integers $a,b$. \vs \\
First observe that $\lim_{r \to 1} f(r)=+\infty$. In fact, given any integer $N>0$, since $\lim_{r \to 1} \sum_{n=0}^N r^{2^n} = N+1$, we can find a sufficiently small $\delta>0$ such that $1-\delta<r<1$ implies $f(r)>\sum_{n=0}^N r^{2^n}>N$. This can be interpreted as saying that the radial limit of $f$ at $1$ (the $2^0$-th root of unity) is infinite. \vs 

Now the definition of $f$ shows that $f(     z^2)=z^2+z^4+z^8+\cdots=f(z)-z$, or $f(z)=z+f(z^2)$. Using this relation, we see that  if the radial limit of $f$ at the $2^n$-th roots of unity is infinite, then the radial limit of $f$ at the $2^{n+1}$-st roots of unity is also infinite. It follows inductively that the radial limit of $f$ at every $z \in \TT$ for which $z^{2^n}=1$ for some $n \geq 0$ must be infinite. All such points are singular points of $f$ and they form a dense subset of $\TT$. Since the singular set is closed, it follows that every point of $\TT$ is singular, i.e., $\TT$ is the natural boundary of $f$. \\ 

\noindent
{\bf Problem 4.} Fix $\alpha>0$ and let $f(z)=\sum_{n=0}^{\infty} 2^{-n\alpha} z^{2^n}$. Show that \vs
\begin{enumerate}[leftmargin=*]
\item[(i)]
The power series has radius of convergence $1$, so by Hadamard's gap theorem, $\TT$ is the natural boundary of $f \in \OO(\DD)$. \vs \\
We have $f(z)=\sum_{k=0}^{\infty} a_k \, z^k$, where $a_k=2^{-n\alpha}$ if $k=2^n$ for some $n \geq 0$ and $a_k=0$ otherwise. Hence, 
$$
\limsup_{k \to \infty} |a_k|^{1/k} = \limsup_{n \to \infty} \ (2^{-n\alpha})^{2^{-n}} = \limsup_{n \to \infty} \ 2^{-n2^{-n} \alpha} = 2^0 =1,
$$
where we have used the fact that $\lim_{n \to \infty} n2^{-n}=0$. It follows that the radius of convergence of the power series is $1$. By Hadamard's gap theorem (Theorem 10.9 with $m_n=2^n$), $\TT$ is the natural boundary of $f$. \vs  

\item[(ii)] 
$f$ has a continuous extension to the closed unit disk $\ov{\DD}$. Moreover, if $\alpha>1$ then $f|_\TT$ is differentiable. \vs \\
Since $|2^{-n\alpha} z^{2^n}| \leq 2^{-n\alpha}$ for $|z| \leq 1$ and $\sum 2^{-n\alpha}$ converges, the Weierstrass $M$-test shows that the power series converges uniformly on $\ov{\DD}$. In particular, it defines a continuous function on $\ov{\DD}$. \vs

The power series representation $f'(z)=\sum_{n=0}^{\infty} 2^{n(1-\alpha)} z^{2^n-1}$ is valid in $\DD$. We have $|2^{n(1-\alpha)} z^{2^n-1}| \leq 2^{n(1-\alpha)}$ for $|z| \leq 1$ and if $\alpha>1$, $\sum 2^{n(1-\alpha)}$ converges. Hence by the Weierstrass $M$-test the power series of $f'$ converges uniformly on $\ov{\DD}$. This proves that if $\alpha>1$ the restriction $f|_\TT$ is differentiable and its derivative is $f'|_\TT$.  
\end{enumerate}

\vs  

\noindent
{\it Comment.} It can be shown that when $0<\alpha\leq 1$ the restriction $f|_\TT$ is a nowhere differentiable curve. Compare the following graphs which render close approximations to this curve for $\alpha=1.3$ (left) and $\alpha=0.8$ (right): \vs

\begin{minipage}{0.45\textwidth}
 	\begin{overpic}[height=6cm]{loop1.pdf}
	\end{overpic}
\end{minipage}
\hfill
\begin{minipage}{0.45\textwidth}
 	\begin{overpic}[height=6cm]{loop2.pdf}
	\end{overpic}
\end{minipage}  

\vs \vs 

\noindent
{\bf Problem 5.} Imitate the proof of Theorem 10.5 to show that every closed subset of $\TT$ is the singular set of some holomorphic function in $\DD$. \vs \\
Let $E$ be a non-empty closed subset of $\TT$. If $E=\{ q_1, \ldots, q_k \}$ is finite, the function $\sum_{n=1}^k 1/(z-q_n) \in \OO(\DD)$ has the singular set $E$. So let us assume $E$ is infinite. Take a dense sequence $\{ q_n \}_{n \geq 1}$ in $E$, making sure that each isolated point of $E$ (if any) appears infinitely often in this sequence. This is possible because $E$ has at most countably many isolated points. For each $n$ take a point $p_n \in \CC \sm E$ such that $|p_n-q_n|<1/n$. Since $E$ is closed, the accumulation points of the sequence $\{ p_n \}_{n \geq 1}$ all belong to $E$. We claim that in fact every $q \in E$ is an accumulation point of $\{ p_n \}$. If $q$ is not an isolated point of $E$, the density gives a subsequence $\{ q_{n_j} \}$ converging to $q$. It follows from $|p_{n_j}-q| \leq |p_{n_j}-q_{n_j}|+|q_{n_j}-q|<1/n_j + |q_{n_j}-q|$ that $p_{n_j} \to q$ as $j \to \infty$. On the other hand, if $q$ is an isolated point of $E$, by the construction it appears infinitely often in the sequence $\{ q_n \}$, so there is a subsequence $\{ q_{n_j} \}$ taking the constant value $q$. It follows from $|p_{n_j}-q|=|p_{n_j}-q_{n_j}|<1/n_j$ that $p_{n_j} \to q$ as $j \to \infty$. This proves the claim. \vs

Now by the Weierstrass product theorem for general open sets (Theorem 8.25) there is an $f \in \OO(\CC \sm E)$ which vanishes precisely along $\{ p_n \}$. The restriction of $f$ to $\DD$ has $E$ as its singular set. Clearly every point of $\TT \sm E \subset \CC \sm E$ is a regular point of $f$. If some $q \in E$ were regular, we could extend $f$ holomorphically to an open disk $D$ centered at $q$. Then $q$ would be an accumulation point of the zeros of $f$, so by the identity theorem $f=0$ in $D$ hence in $\CC \sm E$.  \\  

\noindent
{\bf Problem 6.} According to a theorem of Vivanti and Pringsheim (1893-1894), if $f(z)=\sum_{n=0}^\infty a_n \, z^n$ has radius of convergence $1$ and $a_n \geq 0$ for all $n$, then $1 \in \TT$ is a singular point of $f$. Prove this result by completing the following outline: Assume $f$ extends holomorphically to a neighborhood of $1$. Then the power series of $f$ centered at $\tfrac{1}{2}$ would converge in the disk $\DD(\tfrac{1}{2},\tfrac{1}{2}+\ve)$ for a small $\ve>0$. Hence  $f(z)=\sum b_n (z-\tfrac{1}{2})^n$ for $|z-\tfrac{1}{2}|<\frac{1}{2}+\ve$, where $b_n=\tfrac{1}{n!} f^{(n)}(\tfrac{1}{2})$ can be expressed as an infinite series involving the $a_n$. Substitute this expression for $b_n$ and switch the order of summation to verify that $f(x)=\sum a_n \, x^n$ for real $1<x<1+\ve$, which would be a contradiction. \vs \\
Let us follow the suggested outline. Assume by way of contradiction that $1$ is regular and extend $f$ holomorphically to an open disk $B$ centered at $1$. For small $\ve>0$ the disk $\DD(\tfrac{1}{2},\tfrac{1}{2}+\ve)$ is contained in $\DD \cup B$, so $f$ has a power series representation of the form $f(z)=\sum_{n=0}^{\infty} b_n (z-\tfrac{1}{2})^n$ for $|z-\tfrac{1}{2}|<\frac{1}{2}+\ve$. Here 
\begin{align*}
b_n & = \frac{1}{n!} f^{(n)}\Big(\frac{1}{2}\Big) \\
& = \frac{1}{n!} \sum_{k=n}^{\infty} k(k-1)\cdots (k-n+1) \ a_k \, \Big( \frac{1}{2} \Big)^{k-n} \\
& = \frac{1}{n!} \sum_{k=n}^{\infty} \frac{k!}{(k-n)!} \ a_k \, \Big( \frac{1}{2} \Big)^{k-n} = \sum_{k=n}^{\infty} {k \choose n} \ a_k \, \Big( \frac{1}{2} \Big)^{k-n}. 
\end{align*}
It follows that for real $1<x<1+\ve$,
$$
f(x) = \sum_{n=0}^{\infty} b_n \Big(x-\frac{1}{2} \Big)^n = \sum_{n=0}^{\infty} \sum_{k=n}^{\infty} {k \choose n} \ a_k \, \Big( \frac{1}{2} \Big)^{k-n} \Big(x-\frac{1}{2} \Big)^n.
$$
Since all terms in this double series are positive (this is where we use the assumption $a_k \geq 0$), we can switch the order of summation to obtain 
$$
f(x) = \sum_{k=0}^{\infty} \ \Big[ \underbrace{\sum_{n=0}^k {k \choose n} \Big( \frac{1}{2} \Big)^{k-n} \Big(x-\frac{1}{2} \Big)^n}_{\text{binomial expansion of} \ (x-\tfrac{1}{2}+\tfrac{1}{2})^k} \Big] \ a_k = \sum_{k=0}^{\infty} a_k x^k.
$$
This is a contradiction because the power series $\sum a_k z^k$, having the radius of convergence $1$, must diverge at every $z$ with $|z|>1$.   

\end{document}

