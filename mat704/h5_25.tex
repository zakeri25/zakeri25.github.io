\documentclass[12pt]{amsart}
\bibliographystyle{amsalpha}
\usepackage{mathpazo,amssymb,amscd,epsfig,latexsym,graphicx,eucal}

\setlength\textheight{8in} 
\setlength\textwidth{6in}
\setlength\oddsidemargin{0.2in} 
\setlength\evensidemargin{0.2in}

\newcommand{\ds}{\displaystyle}
\newcommand{\diam}{\operatorname{diam}}
\newcommand{\area}{\operatorname{area}}
\newcommand{\dist}{\operatorname{dist}}
\newcommand{\len}{\operatorname{length}}
\newcommand{\ord}{\operatorname{ord}}
\newcommand{\myint}{\operatorname{int}}
\newcommand{\Aut}{\operatorname{Aut}}
\newcommand{\wtl}{\widetilde}
\newcommand{\wht}{\widehat}
\newcommand{\ve}{\varepsilon}
\newcommand{\es}{\emptyset}
\newcommand{\sm}{\smallsetminus}
\newcommand{\bd}{\partial}
\newcommand{\Chat}{\widehat{\Bbb C}}
\newcommand{\myre}{\operatorname{Re}}
\newcommand{\myim}{\operatorname{Im}}
\newcommand{\ov}{\overline}
\newcommand{\io}{\iota}
\newcommand{\con}{\operatorname{const.}}
\newcommand{\res}{\operatorname{res}}
\newcommand{\OO}{{\mathcal O}}
\newcommand{\MM}{{\mathcal M}}
\newcommand{\CC}{{\mathbb C}}
\newcommand{\PP}{{\mathbb P}}
\newcommand{\RR}{{\mathbb R}}
\newcommand{\HH}{{\mathbb H}}
\newcommand{\TT}{{\mathbb T}}
\newcommand{\II}{{\mathbb I}}
\newcommand{\ZZ}{{\mathbb Z}}
\newcommand{\NN}{{\mathbb N}}
\newcommand{\DD}{{\mathbb D}}
\newcommand{\QQ}{{\mathbb Q}}
\newcommand{\vs}{\vspace{2mm}}
\newcommand{\vr}{\varrho}
\newcommand{\wind}{{\sf W}}
\newcommand{\iso}{\stackrel{\cong}{\longrightarrow}}
\newcommand{\Cbar}{\overline{\mathbb  C}}
\newcommand{\Cstar}{{\mathbb  C}^*}
\newcommand{\Dstar}{{\mathbb  D}^*}
\newcommand{\Sen}{{\mathbb  S}^1}

%Discourage hyphenation:
\hyphenpenalty=5000 \tolerance=1000

\thispagestyle{empty}

\input{figs}

\begin{document}
\begin{center}
{\bf \large Math 704 Problem Set 5} \vs \\
{\bf due Monday 3/17/2025} \vs \vs
\end{center}

\noindent
{\bf Problem 1.} Let $K$ be a compact subset of the unit circle $\TT$. If $K=\TT$, every polynomial $P$ satisfies $|P(0)| \leq \sup_{z \in K} |P(z)|$ by the maximum principle. If $K \neq \TT$, this can fail dramatically. Show that in this case for every $\ve>0$ there is a polynomial $P$ with $P(0)=1$ such that $\sup_{z \in K} |P(z)|<\ve$. (Hint: Use Runge's theorem on $K$ for a suitable $f \in \OO(\Cstar)$.) \vs

\noindent
{\bf Problem 2.} Show that there is a sequence $\{ P_n \}$ of polynomials such that
$$
\lim_{n \to \infty} P_n(z)= \begin{cases} \ \ \ 1 & \quad \text{if} \ \myim(z)>0 \\
\ \ \ 0 & \quad \text{if} \ \myim(z)=0 \\
-1 & \quad \text{if} \ \myim(z)<0.
\end{cases}
$$
Can you achieve the additional property
$$
|P_n(z)| \leq 1 \qquad \text{for every} \ z \in \DD \ \text{and} \ n \geq 1? \vs
$$

\noindent
{\bf Problem 3.} Is there a sequence of polynomials which tends to $0$ compactly in the upper half-plane but does not have a limit at any point of the lower half-plane? \vs  

\noindent
{\bf Problem 4.} Deduce Mittag-Leffler's Theorem 9.4 for an open set $U$ from Runge's Theorem 9.18 by completing the following outline: Let $\es = K_0 \subset K_1 \subset K_2 \subset \cdots$ be a nice exhaustion of $U$. For $n \geq 1$, let $Q_n$ be the finite sum of the principal parts $P_k(1/(z-z_k))$ over all $k$ such that $z_k \in K_n \sm K_{n-1}$. For each $n \geq 2$, find a rational function $R_n$ with poles outside $U$  such that $|Q_n-R_n| \leq 2^{-n}$ on $K_{n-1}$. Show that $f=Q_1+\sum_{n=2}^{\infty} (Q_n-R_n)$ is a meromorphic function in $U$ with the principal part $P_k(1/(z-z_k))$ at each $z_k$, and with no other poles. \vs 

\noindent
{\bf Problem 5.} Let $K \subset \CC$ be compact and connected. Show that every connected component of $\Chat \sm K$ is simply connected. \vs

\noindent
{\bf Problem 6.} Let $U,V \subset \CC$ be simply connected domains with $U \cap V \neq \es$. Show that every connected component of $U \cap V$ is simply connected. (Hint for both problems 5 and 6: Use the fact that simple connectivity of a domain $\Omega$ is equivalent to $\Chat \sm \Omega$ being connected, or to $H_1(\Omega)=0$.) 

\end{document}
