\documentclass[12pt]{amsart}
\bibliographystyle{amsalpha}
\usepackage{mathpazo,amssymb,amscd,epsfig,latexsym,graphicx,eucal}

\setlength\textheight{8in} 
\setlength\textwidth{6in}
\setlength\oddsidemargin{0.2in} 
\setlength\evensidemargin{0.2in}

\newcommand{\ds}{\displaystyle}
\newcommand{\diam}{\operatorname{diam}}
\newcommand{\area}{\operatorname{area}}
\newcommand{\dist}{\operatorname{dist}}
\newcommand{\ord}{\operatorname{ord}}
\newcommand{\myint}{\operatorname{int}}
\newcommand{\Aut}{\operatorname{Aut}}
\newcommand{\wtl}{\widetilde}
\newcommand{\wht}{\widehat}
\newcommand{\ve}{\varepsilon}
\newcommand{\es}{\emptyset}
\newcommand{\sm}{\smallsetminus}
\newcommand{\bd}{\partial}
\newcommand{\Chat}{\widehat{\Bbb C}}
\newcommand{\Real}{\operatorname{Re}}
\newcommand{\Image}{\operatorname{Im}}
\newcommand{\ov}{\overline}
\newcommand{\io}{\iota}
\newcommand{\con}{\operatorname{const.}}
\newcommand{\res}{\operatorname{res}}
\newcommand{\OO}{{\mathcal O}}
\newcommand{\MM}{{\mathcal M}}
\newcommand{\CC}{{\mathbb C}}
\newcommand{\PP}{{\mathbb P}}
\newcommand{\RR}{{\mathbb R}}
\newcommand{\HH}{{\mathbb H}}
\newcommand{\TT}{{\mathbb T}}
\newcommand{\II}{{\mathbb I}}
\newcommand{\ZZ}{{\mathbb Z}}
\newcommand{\NN}{{\mathbb N}}
\newcommand{\DD}{{\mathbb D}}
\newcommand{\QQ}{{\mathbb Q}}
\newcommand{\vs}{\vspace{2mm}}
\newcommand{\vr}{\varrho}
\newcommand{\iso}{\stackrel{\cong}{\longrightarrow}}
\newcommand{\Cbar}{\overline{\mathbb  C}}
\newcommand{\Cstar}{{\mathbb  C}^*}
\newcommand{\Dstar}{{\mathbb  D}^*}
\newcommand{\Sen}{{\mathbb  S}^1}

%Discourage hyphenation:
\hyphenpenalty=5000 \tolerance=1000

\thispagestyle{empty}

\input{figs}

\begin{document}
\begin{center}
{\bf \large Math 704 Problem Set 3} \vs \\
{\bf due Monday 2/24/2025} \vs \vs
\end{center}

\noindent
{\bf Problem 1.} In Example 9.2 we constructed a meromorphic function $f$ in $\CC$ with the principal part $1/(z-n)$ at every $n \in \ZZ$, and with no other poles:
$$
f(z) = \frac{1}{z} + \sum_{n \in \ZZ \sm \{ 0 \}} \left(  \frac{1}{z-n} +\frac{1}{n} \right).
$$
Show that in fact $f(z)=\pi \cot(\pi z)$. (Hint: Use problem 4 homework 1  to verify that $f'(z)=-\pi^2/\sin^2(\pi z)$, so $f(z)=\pi \cot(\pi z)+C$ for some constant $C$. Compare the Laurent expansions of both sides near the origin to deduce $C=0$.)\vs

\noindent
{\bf Problem 2.} Prove that
$$
f(z)=\sum_{n=-\infty}^{\infty} \frac{1}{z^3-n^3}
$$
defines a meromorphic function in $\CC$. Identify the poles and principal parts of $f$. (Hint: For convergence, follow the usual $M$-test routine: Fix an arbitrary $r>0$ and bound $1/|z^3-n^3|$ from above for $|z|<r$ and $|n| \geq 2r$.) \vs 

\noindent
{\bf Problem 3.} Construct, using an explicit infinite series, a meromorphic function in $\CC$ with the principal part $1/(z-\log n)$ at $\log n$ for every integer $n \geq 1$, and with no other poles. (Hint: Imitate the proof of Mittag-Leffler's Theorem 9.1. For $n \geq 2$ take $Q_n(z)$ to be the degree $n$ Taylor polynomial of $1/(z-\log n)$ centered at $0$.)  

\end{document}
